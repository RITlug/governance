\documentclass[a4paper]{article}

\usepackage[english]{babel}
\usepackage[utf8x]{inputenc}
\usepackage[margin=1in]{geometry}

\usepackage{titlesec}
\usepackage{enumitem}
\titleformat{\section}{\large\scshape}{\thesection}{0em}{}
\titleformat{\subsection}{\normalfont\scshape}{\thesubsection}{0em}{}
\renewcommand*\thesection{Function \Roman{section}: }
\renewcommand*\thesubsection{Subroutine \Roman{subsection}: }

\newcommand{\pagetitle}[2]{
	\begin{center}
	\LARGE {#1} \\[0.0625in]
	\large {#2}
	\end{center}
}

\newlist{subroutines}{enumerate}{1}
\setlist[subroutines]{label = Subroutine \Roman*:, leftmargin = 8em, labelsep = 0.5em, labelwidth = 7.5em, align = left}

\begin{document}

\pagetitle{RITlug Server Use Policy}{Updated March 17, 2014}

\section{Access Privileges}

\begin{itemize}
\item Members of the Executive Board and those deemed trustworthy by the Executive Board may hold access to elevated privileges (e.g. sudo).
\item Those with access to use elevated privileges shall not use their access to violate the privacy of other members without probable cause as decided by either the Executive Board or the Institute.
\item An installation or change in accesibility of software or services must be approved by the Executive Board, who shall decide the appropriate requirements for the documentation of the new software or services.
\item Those with access to elevated privileges may modify or move server configurations or data as necessary, without violating any other policies.
\item Accounts may be requested from the Executive Board and will be created at the Executive Board's discretion.
\end{itemize}

\section{Resource Use}

\subsection{Storage}
\begin{itemize}
\item Users may use storage space to store content as long as it does not violate club or Institute policies.
\item Users may not use an excessive amount of storage space, as decided by the Executive Board.
\item Users who store self-produced content or code shall retain the rights of said content and do not grant RITlug any rights for its use unless otherwise stated.
\end{itemize}

\subsection{CPU/Memory}
\begin{itemize}
\item Users may use the RITlug servers for their own research, compilation, and arbitrary tasks, but should not do anything that may require a substantial amount of resources without first obtaining permission from the Executive Board.
\item Users should monitor their running software and tasks so that problems can be promptly resolved if they occur.
\end{itemize}

\section{Service Use}

\subsection{Task Scheduling}
\begin{itemize}
\item Users may not create scheduled system tasks without obtaining permission from the Executive Board.
\item Users may schedule their own personal tasks, as long as said tasks do not consume an excessive amount of resources and follow all given policies.
\end{itemize}

\subsection{Servers}
\begin{itemize}
\item Users may publically serve content through the Internet as long as doing so does not violate any other policies.
\item Users may not publically serve content for which they do not own the proper permissions and copyrights.
\item Users shall not use the servers to publicise content that the Executive Board deems offensive or otherwise in bad taste.
\item The Executive Board may vote to remove, revoke, or disallow server use, content distribution, or content storage at any time.
\end{itemize}

\section{Virtual Machine Use}
\begin{itemize}
\item Virtual machines are governed by the previously stated policies.
\item The Owner of a virtual machine shall have root access to their virtual machine.
\item The Executive Board may vote to remove any virtual machine at their discretion if they feel that it violates any of the above policies.
\end{itemize}

\section{Removing Accounts/Data}
\begin{itemize}
\item If data which is scheduled to be removed is not deemed overly offensive or in violation of Institute policy, the owner of said data shall be given an advance warning of at least 24 hours to retrieve or back up their data before it is removed.
\item The above policy also applies to removed accounts.
\item Accounts and data shall not be removed without first consulting the Executive Board or an official from the Institute.
\end{itemize}

\end{document}
