\documentclass[a4paper]{article}

\usepackage[english]{babel}
\usepackage[utf8x]{inputenc}
\usepackage[margin=1in]{geometry}

\usepackage{titlesec}
\usepackage{enumitem}
\titleformat{\section}{\large\scshape}{\thesection}{0em}{}
\titleformat{\subsection}{\normalfont\scshape}{\thesubsection}{0em}{}
\renewcommand*\thesection{Function \Roman{section}: }
\renewcommand*\thesubsection{Subroutine \Roman{section}.\Roman{subsection}: }

\newlist{subroutines}{enumerate}{1}
\setlist[subroutines]{label = Subroutine \Roman*:, leftmargin = 8em, labelsep = 0.5em, labelwidth = 7.5em, align = left}
\begin{document}

\pagetitle{RIT Linux Users Group Constitution}{Effective May 06, 2017}

\section{Name}
The name of this organization shall be the RIT Linux Users Group (RITlug).

\section{Derivation of Authority}
RITlug shall recognize that it receives its right to function as an Institute Organization from Student Government of the Rochester Institute of Technology in accordance with The Club Guidelines and The Student Government Constitution.

\section{Purpose}
The purpose of the organization is to advocate, educate and provide a venue for discussion and promotion of the Linux Operating System, Free and Open Source Software, and Technology Culture.

\section{Rights and Responsibilities}
\begin{subroutines}
\item RITlug has the right to pursue any activities to achieve its stated goals and/or purposes, as long as those activities do not violate of any rules and regulations of Student Government and the Rochester Institute of Technology.
\item RITlug shall have at least one member on its Executive Board, or other responsible party present at each regularly scheduled club meeting.
\item RITlug shall strictly adhere to all the rules and regulations of Student Government and the Rochester Institute of Technology.
\item RITlug has the right to obtain a seat on the Club Review Board, in accordance with the policies of the Student Government Club Guidelines.
\end{subroutines}

\section{Membership}
\begin{subroutines}
\item Membership is open to regularly enrolled undergraduates, post graduates, faculty, staff and night students, in good standing with the Institute.
\item Active members shall be defined as those in good standing who have met the qualifications set forth in the Student Government Club Guidelines. Only active members are eligible to vote in elections.
\item A member may resign by submitting the intention in writing to an officer of the Executive Board. It becomes effective upon submission.
\item A member shall be declared inactive if he/she attends fewer than four official club activities within a given academic term. Exception is made for members on co-op, the summer term, and intersession.
\end{subroutines}

\section{Officers}
\begin{subroutines}
\item The duties of the officers in this organization shall be in order of rank: President, Vice-President, Secretary, Treasurer, and Executive Board (Eboard) Member(s). RITlug (and Executive Board) may choose to rename these positions.
\item Qualifications for office shall be to maintain good academic standing as defined by R.I.T. (not on academic probation). In order to be nominated for office, member must have been active in the club during the previous academic term (not including the summer term and intersession), as defined in Membership.
\item Officers shall be elected in the manner provided in the constitution and shall serve one calendar year beginning after elections. Officers are eligible for re-election.
\item No officer may hold more than one office simultaneously, although they may take on the responsibilities of another office if the need is present either during the year or at election time.
\item An officer may resign from office by submitting the intention in writing to the Executive Board. The resignation becomes effective upon submission to the Executive Board. Upon resignation, the Coordinator of club affairs must be informed.
\item The Executive Board may appoint new officers and create new positions as they see fit, by unanimous vote.
\end{subroutines}

\section{Advisors}
\begin{subroutines}
\item There shall be at least one club advisor.
\item Nominations for an advisor shall come from the floor, by any active member, the meeting after the officer election.
\item The election of an advisor shall take place at the next business meeting, a simple majority necessary, in accordance with the elections guidelines set forth in this Constitution.
\item Only R.I.T. staff and faculty are qualified to become an advisor, unless with permission of the Coordinator of Club Affairs and or the Club Review Board.
\item If the club advisor resigns, the Coordinator of Club Affairs must be informed.
\item The duty of the advisor shall be to ensure that the club follows Institute rules and regulations.
\end{subroutines}

\section{Duties of Officers}
\subsection{President}
\begin{subroutines}
	\item Preside over all meetings and activities of the organization, and the Executive Board.
	\item Attend all general club meetings, designating an alternative member to attend meetings in the President's absence.
	\item Appoint all committees, regular or special.
	\item Enforce all internal club policies.
\end{subroutines}
\subsection{Vice-President}
\begin{subroutines}
	\item Take over the duties of the President when needed or in the temporary absence of the President.
	\item Act as an ex-officio member of all committees within the club.
	\item Supervise all financial transactions of the organization and keep all financial records.
	\item Maintain regular contact with the Secretary of Finance of Student Government.
\end{subroutines}
\subsection{Secretary}
\begin{subroutines}
	\item Take over the duties of the Vice President in the temporary absence of the Vice President.
	\item Keep an accurate roll of members and keep the minutes of all meetings.
	\item Be responsible for maintaining and updating all paperwork with Student Government.
	\item Maintain all internal club policies and documentation.
\end{subroutines}
\subsection{Treasurer}
\begin{subroutines}
        \item Handle incoming donations of any form to the club.
        \item Manage the club budget including budget requests and club payments.
        \item Be responsible for the financial upkeep of club property (such as recurring payments on services).
        \item Advise the RITlug Eboard on financial matters, proactively where appropriate and on request of Eboard officers.
\end{subroutines}
\subsection{Project Coordinator}
\begin{subroutines}
        \item Act as liason between project leads and the rest of eboard.
        \item Mediate disputes between project leads and team members.
        \item Encourage members to start and participate in club projects.
        \item Assist project leads in managing their projects.
        \item Assist project leads in establishing project-specific meetings.
        \item Provide general support to project leads and participants.
\end{subroutines}

\section{Executive Board}
\begin{subroutines}
\item The management of this organization shall be vested in the Executive board.
\item The Executive Board shall be composed of the President, Vice-President, Secretary, Advisors, Treasurer, and Executive Board members, or equivalent titles if appropriate.
\item The Executive Board shall meet as needed and report to the members at the following business meetings.
\item The Executive Board shall fill any permanent vacancy of office by appointment, excepting the Presidency.
\item In the case of a permanent vacancy in the office of the Presidency, a special election shall be held in the manner of annual elections as set forth in the constitution.
\item The Executive Board may appoint active club members as additional Executive Board Members under the rank "Eboard Member", temporarily or permamently on an as-needed basis, following with the rules set in the preceeding subroutines.
\end{subroutines}

\section{Meetings}
\begin{subroutines}
\item Meetings shall be called as needed (or at least once a month, with the exception of summer term and intersession) at a time and place determined by the Executive Board.
\item The Annual meeting shall take place during spring term. After the annual meeting annual reports to Student Government shall be presented.
\item Special meetings may be called by a member of the Executive Board upon 24 hours written notice to the members.
\item In order for any business to take place at a meeting, a quorum shall be presumed unless challenged from the floor by a member in good standing.
\end{subroutines}

\section{Record Keeping}
\begin{subroutines}
\item The Executive Board shall maintain and keep updated a record of club history, including Eboard members and notable club happenings.
\item The Executive Board shall maintain meeting notes on Executive Board proceedings. If the Executive Board is not fully staffed, this subroutine may be waived at the discretion of the active board.
\item The Executive Board shall make available on request club history and past records, excluding information deemed confidential, to RIT and active club members.
\item The Executive Board shall maintain and keep updated records on how the club operates and how to manage club property and proceedings.
\end{subroutines}

\section{Elections}
\begin{subroutines}
\item Elections shall take place as needed.
\item Nominations for office shall be made from the floor at the meeting prior to the annual meeting by any member in good standing. A person may be nominated for more than one office.
\item The order of election shall be President, Vice-President, Secretary, Treasurer, and Executive Board Members.
\item Votes shall be cast by secret ballot, with an exception made for write-ins.
\item Elections shall be scheduled no less than one week in advance, during a regularly scheduled club meeting.
\item Elections shall take place no less than three weeks prior to the end of the semester to allow for a transitional period.
\item Should students who are interested but technically not considered active voting members decide to vote, they may cast a write-in vote due by the time of vote. The Executive Board shall reach a consensus on the validity of each write-in on a case-by-case basis.
\item Induction of new officers shall take place at the meeting after elections.
\end{subroutines}

\section{Transfer of Accounts}
\begin{subroutines}
\item The term "Superuser" shall refer to an account with the highest possible level of permissions. In the case of a Linux- or Unix-based operating system, this refers specfically to the user account with ID 1.
\item Superuser passwords for all RITlug-owned devices (e.g. servers) must be given to the new President within a reasonable time period after every election.
\item Any online account which is created for the sole purpose of serving RITlug, not including accounts on the behalf of any one member of RITlug, must be recoverable with the services that are provided by RITlug's status as a club (e.g. \texttt{ritlug@rit.edu}).
\item Any online organisation (e.g. GitHub organisation, subreddit, IRC) which is created to represent RITlug must contain one of the above accounts with superuser privileges.
\end{subroutines}

\section{Impeachment}
\begin{subroutines}
\item Any officer of this organization can be impeached.
\item Impeachment may be initiated by petition, in writing, by 25\% or more of the active members, when presented to the members in a regular meeting.
\item At the next meeting, the accuser (members signing petition) and accuse’ (officer) shall present their case to the members. After both cases are heard, a written vote shall be taken.
\item Conviction of impeachment shall require a 2/3 vote by the members in good standing present, a quorum being present.
\item Conviction of impeachment shall cause removal from office and loss of all privileges thereof.
\item The filling of an office vacated by impeachment shall be by a special election held in the manner of annual elections.
\item The Executive Board reserves the right to remove an officer by unanimous vote, excluding the offending officer, with the consent of the Advisor.
\end{subroutines}

\section{Amendments}
\begin{subroutines}
\item An amendment to this constitution may be proposed by any member in good standing, in writing, at any business meeting. There must be copies of the amendment made for all members.
\item The proposed amendment shall require a 2/3 affirmative vote by a voting quorum.
\item Before the proposed amendment shall become operational it must be approved by the Coordinator of Club Affairs.
\end{subroutines}

\section{By-Laws}
\begin{subroutines}
\item Although it is recommended that written copies be handed out to all members, By-laws of this organization can be proposed verbally at any meeting by any member in good standing, and copies of all official policy will be available online.
\item An immediate vote, after discussion, shall be taken by a show of hands and a simple majority is necessary for passage, a quorum being present.
\item By-Laws do not require the approval of the Department of Club Affairs.
\item There will be no hazing of any kind within this organization in accordance with the New York State Hazing Laws.
\item Records of these bylaws will be held by the Secretary and copies given to the advisor.
\end{subroutines}

\section{Parliamentary Authority}
\begin{subroutines}
\item The parliamentary authority of this organization shall be its Constitution parliamentary authority of all matters not covered in the Constitution or By-Laws as deemed necessary by the Presiding Officer.
\end{subroutines}

\end{document}
